\documentclass[11pt]{article}

\usepackage{times}
\usepackage[english]{babel}

% -----------------------------------------------
% especially use this for you code
% -----------------------------------------------

\usepackage{courier}
\usepackage{listings}
\usepackage{color}
\usepackage{tabularx}
\usepackage{graphicx}

\definecolor{Gray}{gray}{0.95}

\definecolor{mygreen}{rgb}{0,0.6,0}
\definecolor{mygray}{rgb}{0.5,0.5,0.5}
\definecolor{mymauve}{rgb}{0.58,0,0.82}

\lstset{language=C++,
	basicstyle = \normalsize\ttfamily,   % the size and fonts that are used
	tabsize = 2,                    % sets default tabsize
	breaklines = true,              % sets automatic line breaking
	keywordstyle=\color{blue}\ttfamily,
	stringstyle=\color{red}\ttfamily,
	commentstyle=\color{mygreen}\ttfamily,
	numbers=left,
	keepspaces=true,
	showspaces=false,
	showstringspaces=false,
}

\begin{document}

\title{Programming in C/C++ \\
       Exercises set one: class templates
}
\date{\today}
\author{Christiaan Steenkist \\
Jaime Betancor Valado \\
Remco Bos \\
}

\maketitle
\section*{Exercise 1, new matrix}
In this exercise we changed the matrix class to work with templates.
Keep in mind that this would make the header file humongous as the entirety of the template class needs to be in the header.
Since we had all the functions in seperate files anyways we just kind of left them there and only included the destructor.
Just imagine all the functions as being place in the header file as the destructor is with no seperate declaration and definition.

\subsection*{Code listings}
\lstinputlisting[caption = matrix.ih]{src/a1/matrix.ih}
\lstinputlisting[caption = matrix.h]{src/a1/matrix.h}
\lstinputlisting[caption = main.cc]{src/a1/main.cc}

\subsubsection*{matrix files}
\lstinputlisting[caption = matrix1.cc]{src/a1/matrix1.cc}
\lstinputlisting[caption = matrix2.cc]{src/a1/matrix2.cc}
\lstinputlisting[caption = matrix3.cc]{src/a1/matrix3.cc}
\lstinputlisting[caption = matrix4.cc]{src/a1/matrix4.cc}

\lstinputlisting[caption = add.cc]{src/a1/add.cc}
\lstinputlisting[caption = operatoradd1.cc]{src/a1/operatoradd1.cc}
\lstinputlisting[caption = operatoradd2.cc]{src/a1/operatoradd2.cc}
\lstinputlisting[caption = operatoraddis1.cc]{src/a1/operatoraddis1.cc}
\lstinputlisting[caption = operatoraddis2.cc]{src/a1/operatoraddis2.cc}
\lstinputlisting[caption = operatorassign1.cc]{src/a1/operatorassign1.cc}
\lstinputlisting[caption = operatorassign2.cc]{src/a1/operatorassign2.cc}
\lstinputlisting[caption = operatorequal.cc]{src/a1/operatorequal.cc}
\lstinputlisting[caption = operatorfun1.cc]{src/a1/operatorfun1.cc}
\lstinputlisting[caption = operatorfun2.cc]{src/a1/operatorfun2.cc}

\subsubsection*{proxy files}
\lstinputlisting[caption = proxy1.cc]{src/a1/proxy1.cc}
\lstinputlisting[caption = proxyextractcols.cc]{src/a1/proxyextractcols.cc}
\lstinputlisting[caption = proxyextractfrom.cc]{src/a1/proxyextractfrom.cc}
\lstinputlisting[caption = proxyextractrows.cc]{src/a1/proxyextractrows.cc}

\section*{Exercise 2, Member template}
In this exercise we changed one of the methods of the Semaphore class to make it a member template using perfect forwarding.

\subsection*{Code listings}
\lstinputlisting[caption = semaphore.ih]{src/a2/semaphore.ih}
\lstinputlisting[caption = semaphore.h]{src/a2/semaphore.h}
\lstinputlisting[caption = constructor.cc]{src/a2/constructor.cc}
\lstinputlisting[caption = notify.cc]{src/a2/notify.cc}
\lstinputlisting[caption = size.cc]{src/a2/size.cc}
\section*{Exercise 3, custom back inserter}
In this exercise we make a custom class work with the \texttt{back\_inserter} iterator so we can use the \texttt{copy} generic algorithm.

\subsection*{Code listings}
\lstinputlisting[caption = data.ih]{src/a3/data.ih}
\lstinputlisting[caption = data.h]{src/a3/data.h}
\lstinputlisting[caption = main.cc]{src/a3/main.cc}
\lstinputlisting[caption = pushback.cc]{src/a3/pushback.cc}

\section*{Exercise 5, static polymorphism}
We made a static polymorphic class that prints things!

\subsection*{Code listings}
\lstinputlisting[caption = inserter.ih]{src/a5/inserter.ih}
\lstinputlisting[caption = inserter.h]{src/a5/inserter.h}
\lstinputlisting[caption = main.ih]{src/a5/main.ih}
\lstinputlisting[caption = main.h]{src/a5/main.h}
\lstinputlisting[caption = main.cc]{src/a5/main.cc}

\subsubsection*{IntValue}
\lstinputlisting[caption = intconstructor.cc]{src/a5/intconstructor.cc}
\lstinputlisting[caption = intinserter.cc]{src/a5/intinserter.cc}

\subsubsection*{DoubleValue}
\lstinputlisting[caption = doubleconstructor.cc]{src/a5/doubleconstructor.cc}
\lstinputlisting[caption = doubleinserter.cc]{src/a5/doubleinserter.cc}

\section*{Exercise 6, static polymorphism contd.}
Now with more inheritence?

\subsection*{Code listings}
\lstinputlisting[caption = main.ih]{src/a6/main.ih}
\lstinputlisting[caption = main.h]{src/a6/main.h}
\lstinputlisting[caption = main.cc]{src/a6/main.cc}

\subsubsection*{LabelledInt}
\lstinputlisting[caption = labelconstructor.cc]{src/a6/labelconstructor.cc}
\lstinputlisting[caption = labelinserter.cc]{src/a6/labelinserter.cc}

\end{document}